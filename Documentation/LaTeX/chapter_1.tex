%Filename: chapter_1.tex
% Description: Introduction for SP
\chapter{Introduction}

\section{Insert Title here}

With the rapid advancement of computer science and computational linguistics, numerous technologies are now being applied in various real-world situations. Among them is the interdisciplinary subfield known as speech recognition. Also known as automatic/computer speech recognition (ASR) or speech-to-text (STT), it utilizes computers in order to recognize and translate natural spoken language into text.

One of the key applications of automatic speech recognition is to transcribe speech documents such as talks, presentations, lectures, and broadcast news (Furui, et al., 2001, as cited in Furui, et al., 2004). A known challenge in speech transcription is that it can be quite taxing to retrieve and reuse speech documents if they are only recorded as audio. Although high recognition accuracy can be easily obtained for speech read from a text, such as anchor speakers’ broadcast news utterances, technological ability for recognizing spontaneous speech is still limited (Furui, 2003, as cited in Furui, et al., 2004).

About 4500 languages exist in the world, but the majority of languages are spoken by less than 100,000 speakers; only about 150 languages (3\%) have more than 1 Million speakers (Schultz, 2002).

Aklanon, which is often spelled as “Akeanon” by its local writers, is a dialect spoken by people located in the province of Aklan on the island of Panay in the Philippines. It somewhat varies with the dialects of neighboring provinces and islands and it belongs to a family of dialects whose ancestor might be proto-West Visayan, which in turn was a member of the Malayo-Polynesian family of languages, to which such languages as Tagalog and Cebuano belong (De La Cruz \& Zorc, 1968).

Aklanon/Akeanon is a specific language that is mostly exclusive to people who have lived in Aklan therefore most Filipino citizens wouldn’t have familiarity with it. This would prove to be problematic since this would limit possible communications between locals and others. Additionally, there are barely any applications that are able to promote the language and get it out to the public. A speech-to-text recognition system would pave the way for the language to be recognized and appreciated.

This paper presents the development of a speech-to-text system that would be able to recognize Akeanon words with a decent accuracy rating. The system would be made using the open-source speech recognition toolkit “Kaldi”. Kaldi is a speech recognition toolkit written in C++ and licensed under the Apache License v2.0. The system would be developed in a way such that the audio, acoustic and language data would be catered to the Akeanon dialect.


%Problem Statement
\section{Problem Statement}

The creation of an automatic speech-to-text system for the Aklanon language, a language that is not widely used will demonstrate the capabilities of automatic speech recognition toolkits such as “Kaldi”.  Since there is no currently available research on speech-to-text for the Aklanon language, this initial research will serve as an aid or foundation for other speech recognition systems targeting other possible local languages. The existence of speech recognition systems for the Aklanon language will also aid in transcription of language data and therefore easier translation of information and cultural knowledge

%Objectives 
\section{Research Objectives}


\subsection{General Objective}

The general objectives of this research include creating a fully functional speech-to-text system with an adequate level of accuracy when given audio or language data as input. 


\subsection{Specific Objectives}
\begin{itemize}
\item[A.] To review related literature on existing speech-to-text or automatic speech recognition toolkits and compare them with each other to determine which toolkit will be appropriate for the research problem. 

\item[B.] To gather audio or language data from different speakers and convert them to appropriate bit rates and frequencies wherein the toolkit will recognize. 

\item[C.] To develop a system that will utilize the chosen toolkit to recognize simple words from the Aklanon language.
\end{itemize}


%Scope and Limitations
\section{Scope and Limitations of the Research}

The usage of “Kaldi” as opposed to other speech recognition toolkits limits the prototype to only the features that the toolkit has to offer. However, “Kaldi” is more accessible and more supported among any other which makes it ideal as the one to be utilized.

The audio samples used for the speech recognition toolkit were taken from just ten different people. The participants who were recorded for the audio samples used were mostly Aklanon native speakers. This was limited to just speakers of this particular language in order to keep the accuracy of the pronunciations of the words as precise as possible. Moreover, rules on pronunciation for this language could pose difficulties for non-native speakers such as the consonant “e”. 

With regards to the words that the prototype can recognized, it has only been limited to a few simple words. These include numbers, basic directions and common objects. 

\section{Significance of the Research}

Currently, there is no speech recognition software available for the Aklanon language. This study would pave the way for future projects involving Aklanon speech-to-text. The prototype could also serve as a foundation for these future studies as they would expand further on the findings here. 

Considering that Aklanon is similar to its neighboring languages or dialects with only pronunciation and diction being one of its key challenging differences, this opens the door for its respective neighbors to be able to adapt a speech-to-text like this for their own language. This enables access for people to speech recognition technology as more languages would possibly be incorporated in the future.



