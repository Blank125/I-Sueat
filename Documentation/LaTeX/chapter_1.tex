%Filename: chapter_1.tex
\chapter{Introduction}

\section{Insert Title here}

With the rapid advancement of computer science and computational linguistics, numerous technologies are now being applied in various real-world situations. Among them is the interdisciplinary subfield known as speech recognition. Also known as automatic/computer speech recognition (ASR) or speech-to-text (STT), it utilizes computers in order to recognize and translate natural spoken language into text.

One of the key applications of automatic speech recognition is to transcribe speech documents such as talks, presentations, lectures, and broadcast news (Furui, et al., 2001, as cited in Furui, et al., 2004). A known challenge in speech transcription is that it can be quite taxing to retrieve and reuse speech documents if they are only recorded as audio. Although high recognition accuracy can be easily obtained for speech read from a text, such as anchor speakers’ broadcast news utterances, technological ability for recognizing spontaneous speech is still limited (Furui, 2003, as cited in Furui, et al., 2004).

About 4500 languages exist in the world, but the majority of languages are spoken by less than 100,000 speakers; only about 150 languages (3\%) have more than 1 Million speakers (Schultz, 2002).

Aklanon, which is often spelled as “Akeanon” by its local writers, is a dialect spoken by people located in the province of Aklan on the island of Panay in the Philippines. It somewhat varies with the dialects of neighboring provinces and islands and it belongs to a family of dialects whose ancestor might be proto-West Visayan, which in turn was a member of the Malayo-Polynesian family of languages, to which such languages as Tagalog and Cebuano belong (De La Cruz & Zorc, 1968).

Aklanon/Akeanon is a specific language that is mostly exclusive to people who have lived in Aklan therefore most Filipino citizens wouldn’t have familiarity with it. This would prove to be problematic since this would limit possible communications between locals and others. Additionally, there are barely any applications that are able to promote the language and get it out to the public. A speech-to-text recognition system would pave the way for the language to be recognized and appreciated.

This paper presents the development of a speech-to-text system that would be able to recognize Akeanon words with a decent accuracy rating. The system would be made using the open-source speech recognition toolkit “Kaldi”. Kaldi is a speech recognition toolkit written in C++ and licensed under the Apache License v2.0. The system would be developed in a way such that the audio, acoustic and language data would be catered to the Akeanon dialect.
